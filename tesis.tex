\documentclass[12pt]{report}
\usepackage{url}
\usepackage[spanish]{babel}
\usepackage{ucs}
\usepackage[utf8x]{inputenc}
\usepackage{geometry}\geometry{top=5cm,bottom=2cm,left=3cm,right=3cm}
\usepackage{graphicx}
\usepackage{txfonts}
\usepackage{glossaries}
\usepackage[pdfborder={0,0,0}]{hyperref}

\begin{document}

\thispagestyle{empty}

\begin {center}

\includegraphics[scale=1]{logo_fiuba_alta.jpg}

\vspace{3cm}

\textbf{\LARGE Lenguaje de dominio específico embebido para aplicaciones web en un lenguaje multiparadigma funcional - orientado a objetos.}

\vspace{3cm}

\textbf{\Large Tesis de grado en Ingeniería en Informática}

\vspace{2cm}

\end {center}
\vspace{3cm}

Tesista: Mauricio Scheffer

Directora: Rosa Wachenchauzer

Buenos Aires, 2010

\newpage

\tableofcontents

\newpage

\chapter{Introducci\'on}

\chapter{Lenguajes de dominio espec\'ifico}

\section{Introducci\'on}

Usaremos la definición de DSL provista por Van Deursen et al. \cite{van2000domain}: un DSL es un lenguaje de programación o lenguaje de especificación ejecutable que ofrece poder de expresión enfocado en (y generalmente restringido a) un dominio de problemas en particular, a través de notaciones y abstracciones apropiadas.

Clasificaremos los DSLs según su forma de implementación: interno (también llamado embebido) o externo.

\subsection{Caracter\'isticas}

Generalmente los DSLs son \textit{pequeños}, en el sentido de que modelan solamente los conceptos y operaciones del dominio que tratan. Muchos DSLs tienden a ser declarativos, es decir, tienden a definir reglas y estructuras sobre el dominio y luego son transformados a un programa ejecutable por un compilador, intérprete u otro tipo de transformación. 
Otras características de los DSLs incluyen: \cite{van2000domain}

\begin{itemize}
	\item Concisos, precisos
	\item Reutilizables
	\item Encapsulan 
	\item Mantenibles \cite{van1998little}
\end{itemize}



\subsection{Implementaci\'on de DSLs}

\subsubsection{DSLs internos}

Los DSLs embebidos 

\subsubsection{DSLs externos}

\subsection{Programación orientada al lenguaje}

Refs:

\cite{Ward95languageoriented}
\cite{dmitriev2005language}

\chapter{Frameworks web}

\subsection{WebSharper}

WebSharper \cite{websharper} es una plataforma en F\# para desarrollo web. Sus características incluyen: \cite{websharperdocs}

\begin{itemize}
	\item Compilación de F\# a JavaScript.
	\item Implementación de formlets \cite{CLWY08essence} \cite{CLWY08idiomsguide}.
	\item Soporte de comunicación cliente-servidor.
	\item Soporte de una gran parte de la librería standard de F\# in .NET en el cliente.
	\item Programación type-safe 
	\item Facilidades para incorporar código JavaScript y librerías JavaScript externas.
	\item Facilidades para manejo de recursos y dependencias para CSS, imágenes, etc.
	\item Integración con ASP.NET
	\item Integración con Microsoft Visual Studio.
\end{itemize}

Como se ve de la lista de características mencionadas, el principal objetivo de WebSharper es la creación de aplicaciones RIA (rich internet applications) \cite{busch2009rich} y no tanto sitios web en general, en forma similar a otras plataformas como Adobe Flash \cite{flash} o Microsoft Silverlight \cite{silverlight}, pero sin requerir un plugin especial. En cambio, hace hincapié en la generación de código JavaScript a partir de código F\#, en forma similar a F\# Web Tools \cite{petricek-client}, Script\# \cite{scriptsharp} o Google Web Toolkit \cite{gwt}. Algunas ventajas de este enfoque son:

\begin{itemize}
	\item Rápida integración de código del cliente (JavaScript)
	\item Programación de toda la aplicación en un único lenguaje y ambiente.
	\item Típicamente disminuyen el tiempo de respuesta (latencia) en la comunicación cliente-servidor, ya que reducen la cantidad necesaria de datos a transferir.
\end{itemize}

Desventajas:

\begin{itemize}
	\item Requieren un paso adicional de compilación.
	\item Generan una gran cantidad de código JavaScript que debe ser transferido inicialmente al cliente.
	\item Incrementan el acoplamiento de las capas naturalmente desacopladas de una aplicación web.
	\item Al no ajustarse totalmente al modelo HTTP, dificultan ciertas funcionalidades básicas, como el soporte del botón ``Volver`` de los browsers.
\end{itemize}

\chapter{Hacia un DSL para desarrollo web en un lenguaje funcional}

Para diseñar el DSL propuesto en este trabajo, empecemos con la metodología propuesta en \cite{van1998little}:

\begin{itemize}
	\item Identificar el dominio del problema a tratar.
	\item Recopilar todo el conocimiento relevante en dicho dominio.
	\item Resumir este conocimiento a sus nociones semánticas y operaciones fundamentales.
	\item Construir una librería que implemente estas nociones semánticas y operaciones
	\item Diseñar un DSL que describa en forma concisa las aplicaciones en este dominio.
	\item Diseñar e implementar un compilador que convierta los programas escritos en DSL a una secuencia de llamadas a la librería implementada.
\end{itemize}

Nuestro dominio a tratar será el de las ``aplicaciones web``. Esto significa que el experto del dominio, el usuario final del DSL, será un programador web. Con el objetivo de maximizar la reusabilidad del DSL, no restringiremos la clase de aplicaciones target del DSL sino sólo su arquitectura de base (web). 

Elegimos la modalidad de DSL embebido para implementar nuestro DSL por las ventajas mencionadas anteriormente. El lenguaje host será F\# por sus facilidades para programación orientada al lenguaje \cite{syme2007expert} \cite{pickering2007foundations} \cite{smith-lop} \cite{petricek-lop}.

Diseñaremos nuestro DSL sobre la infraestructura de ASP.NET MVC \cite{mvc}, esta infraestructura ya encapsula el conocimiento del dominio, sus conceptos y operaciones. 
Como mencionamos anteriormente, una aplicación web se compone de varias tecnologías disímiles débilmente acopladas. Diseñaremos nuestro DSL con el objetivo de orquestrar las operaciones a ejecutar en el servidor, delegando la generación de la vista (HTML / CSS) a otros frameworks. ASP.NET MVC permite reemplazar fácilmente el generador de vistas \cite{mvcrender}, por lo cual nuestra solución podría usarse con generadores como Web Forms, Spark \cite{spark}, Brail \cite{brail} o Hasic \cite{hasic}. A los efectos de representar todos los aspectos de una aplicación en un único lenguaje, utilizaremos Wing Beats \cite{wingbeats} como generador de vistas en los ejemplos y aplicación de referencia.
También delegaremos otro componente que a veces se incluye en un framework web: el acceso a datos.

\addcontentsline{toc}{chapter}{Referencias}
\bibliographystyle{plain}
\bibliography{tesis}

\end{document}