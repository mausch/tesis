\documentclass[12pt]{article}
\usepackage{url}
\usepackage[spanish]{babel}
\usepackage{ucs}
\usepackage[utf8x]{inputenc}
\usepackage{geometry}\geometry{top=5cm,bottom=2cm,left=3cm,right=3cm}
\usepackage{graphicx}
\usepackage{txfonts}
\usepackage{glossaries}
\usepackage[pdfborder={0,0,0}]{hyperref}

\makeglossaries

\newglossaryentry{dsl}{name={Lenguaje de dominio espec\'ifico (DSL)}, description={lenguaje de programación dedicado a un dominio de problemas limitado y definido.}}
\newglossaryentry{embedded}{name={Lenguaje embebido (\textit{embedded language})}, description={se denomina lenguaje embebido aquel que se incorpora directamente en un código fuente escrito en otro lenguaje (denominado lenguaje \textit{host}). En relación al lenguaje \textit{host}, el lenguaje embebido se denomina lenguaje \textit{guest}.}}
\newglossaryentry{edsl}{name={EDSL (embedded domain-specific language)}, description={lenguaje de dominio específico embebido en un lenguaje host.}}
\newglossaryentry{ast}{name={\'{A}rbol de sintaxis abstracta (abstract syntax tree o AST)}, description={estructura sintáctica de un código fuente representada en un árbol, donde cada nodo representa un token o partícula sintáctica. Convertir el código fuente de un AST es uno de los primeros pasos de un compilador.}}
\newglossaryentry{functional}{name={Programaci\'on funcional}, description={paradigma de programación que tiende a modelar los procesos de computación como funciones matemáticas. A diferencia del paradigma imperativo, evita los estados mutables.}}
\newglossaryentry{multiparadigm}{name={Lenguaje de programación multiparadigma}, description={lenguaje de programación que soporta dos o más paradigmas (funcional, imperativo, orientado a objetos, lógico, etc).}}

\newglossaryentry{fsharp}{name={F\#}, description={lenguaje de programación creado por Microsoft. Fuertemente influenciado por OCaml (otro dialecto de la familia de lenguajes ML), es también un lenguaje multiparadigma ya que abarca los paradigmas funcional e imperativo orientado a objetos. Es multiplataforma: los programas escritos en F\# corren sobre .NET (Windows) como en Mono (GNU/Linux, BSD, Solaris y otros sistemas operativos).}}

\newglossaryentry{ruby}{name={Ruby}, description={lenguaje de programación multiparadigma, aunque principalmente orientado a objetos. Sistema de tipos dinámico.}}
\newglossaryentry{haskell}{name=Haskell, description={lenguaje de programación funcional puro, con fuerte sistema de tipos estático.}}
\newglossaryentry{groovy}{name=Groovy, description={lenguaje de programación orientado a objetos, de tipos dinámicos, que corre en JVM (Java Virtual Machine). Similar a Python y Ruby.}}

\newglossaryentry{clojure}{name=Clojure, description={lenguaje de programación, dialecto de Lisp que corre en JVM.}}

\newglossaryentry{repl}{name={REPL (read-eval-print loop)}, description={ambiente de programación interactivo que ejecuta un ciclo (loop) donde en primer lugar lee (read) la instrucción ingresada por el programador, inmediatamente la evalúa (eval), compilando o interpretando la expresión según sea necesario y finalmente muestra inmediatamente (print) el resultado de evaluar la expresión.}}

\begin{document}
\thispagestyle{empty}

\begin {center}

\includegraphics[scale=1]{logo_fiuba_alta.jpg}


\vspace{3cm}

\textbf{\Large Propuesta de tesis de grado en Ingeniería en Informática}

\vspace{3cm}

\textbf{\Large Lenguaje de dominio específico embebido para aplicaciones web en un lenguaje multiparadigma funcional - orientado a objetos.}

\vspace{2cm}

\end {center}
\vspace{3cm}

Alumno: Mauricio Scheffer

Directora: Rosa Wachenchauzer

Buenos Aires, 2010

\newpage

\tableofcontents

\newpage

\section{Introducción}

\subsection{Lenguajes de dominio específico}

Los \glslink{dsl}{lenguajes de dominio específico (DSL)} permiten una sintaxis simplificada para resolver problemas dentro de un dominio limitado y bien definido. Se usan por lo general en circunstancias donde un lenguaje de propósito general introduciría una complejidad accidental no deseada en la resolución del problema. 

Un DSL bien diseñado facilita la tarea del programador ofreciéndole construcciones sintácticas orientadas al dominio que tratan, al mismo tiempo permaneciendo flexible para que el programador pueda extenderlo o de alguna manera incorporar código propio para resolver tareas complejas para el cual el DSL no fue originalmente creado (siempre y cuando sea dentro del dominio definido).

Un DSL va más allá de una librería en que presenta al programador una interfaz orientada al lenguaje, es decir, construcciones sintácticas específicas.

La práctica extendida de crear una aplicación entera basada en distintos DSLs se denomina “programación orientada al lenguaje” (Language-oriented programming) \cite{Ward95languageoriented} \cite{dmitriev2005language}. Ciertos lenguajes, como Lisp, se prestan naturalmente a este estilo de programación.

Uno de los DSLs más difundidos actualmente es SQL (Structured Query Language), que tiene construcciones sintácticas específicas orientadas a manejar bases de datos relacionales.

\subsubsection{DSLs externos}

Los DSLs externos se implementan definiendo una sintaxis y luego generando un parser y lexer para procesar el código fuente y generar un \glslink{ast}{AST} (abstract syntax tree). La ventaja principal de esta modalidad de DSL es la mayor flexibilidad para definir la sintaxis y la independencia del lenguaje host sobre el cual está implementado.

\subsubsection{DSLs internos}

Los DSLs internos o \glslink{embedded}{embebidos} se implementan aprovechando directamente la sintaxis y compilador del lenguaje host. La ventaja de esta modalidad es una mayor facilidad para implementar y ejecutar el DSL. Los usuarios no necesitan aprender una sintaxis nueva.

\subsection{Frameworks Web}

Una típica aplicación web tiene una arquitectura naturalmente desacoplada, con una variedad de tecnologías en cada capa: HTML y CSS en la capa de presentación (browser), JavaScript a veces colaborando en la capa de presentación y otras veces implementando parte de la lógica del programa dentro del browser, y finalmente el código de servidor. El protocolo usado para comunicar cliente y servidor es HTTP (Hypertext Transfer Protocol).

Un framework web tiene como objetivo unificar todas estas tecnologías y capas dentro de una aplicación y ambiente de programación, ofreciendo una estructura coherente al programador pero al mismo sin crear un excesivo acoplamiento de estas capas. Algunas de las características que un framework web ofrece al programador para su aplicación son: seguridad, mapeo de URLs, acceso a bases de datos, configuración, cache, sistema de plantillas para generación de HTML. No todos los frameworks web implementan todas estas características, algunos las delegan a la infraestructura de la plataforma u otro framework externo.


\section{Estado del arte}

En la actualidad existen literalmente decenas de frameworks para desarollo web, en una multitud de plataformas y lenguajes: Ruby, Python, Java, .NET, Haskell... Cada uno diseñado tratando de explotar las particularidades del lenguaje/plataforma sobre el que está implementado. Así, por ejemplo, los frameworks en Ruby tienden a aprovechar las facilidades de metaprogramación del lenguaje mientras que en Haskell se chequea todo estáticamente aprovechando el rico sistema de tipos.

Más allá de estas particularidades, una arquitectura usada por muchos frameworks es la MVC (Model View Controller), ya que permite una buena separación de responsabilidades entre el modelo de datos, la interfaz del usuario y la lógica del programa. Esto redunda en un menor acoplamiento y mayor cohesión, independientemente de la plataforma o del paradigma del lenguaje.

Por otra parte, la aplicación de DSLs al diseño de frameworks web ya probó ser un camino viable y conveniente, si bien todavía no es de uso masivo. Algunos ejemplos son Sinatra \cite{sinatra} (\gls{ruby}), Happstack \cite{happstack} (\gls{haskell}), Compojure \cite{compojure} (\gls{clojure}) y GroovyRestlet \cite{groovyrestlet} (\gls{groovy}).

En \glslink{fsharp}{F\#} existen varios frameworks web:

\begin{itemize}
	\item Ya que F\# es un lenguaje de primer nivel en .NET, cualquier framework que funcione en C\# / VB.NET funcionará también en F\#, por ejemplo:
	\begin{itemize}
		\item ASP.NET Web Forms \cite{webforms}
		\item ASP.NET MVC \cite{mvc}
		\item Castle MonoRail \cite{monorail}
		\item Bistro \cite{bistro} (que tiene algunas facilidades específicas para F\# \cite{bistroext})
		\item OpenRasta \cite{openrasta}
		\item FubuMVC \cite{fubumvc}
	\end{itemize}
	\item Los frameworks específicos para F\# son:
	\begin{itemize}
		\item WebSharper \cite{websharper}
		\item F\# Web Tools \cite{fswebtools}
		\item \#light \cite{sharplight}
	\end{itemize}
\end{itemize}

\section{Objetivo de la tesis}

Los frameworks web mencionados anteriormente son en su mayoría orientados a objetos. Si bien F\# es un lenguaje \glslink{multiparadigm}{multiparadigma}, soportando el paradigma orientado a objetos tanto como el funcional, sus raíces en ML hacen que en la práctica sea más funcional que orientado a objetos.

Por lo tanto, proponemos como objetivo de esta tesis:
\begin{enumerate}
	\item Explorar los conceptos que debería tener un framework web para aprovechar las características funcionales de F\#.
	\item Comparar estas características con las de los frameworks web ya existentes mencionados anteriormente.
	\item Utilizando la información obtenida de los pasos anteriores, desarrollar un framework funcional para F\#. 
\end{enumerate}

\section{Plan de trabajo}

Estas serán las etapas del trabajo a desarrollar:

\begin{enumerate}
	\item Análisis de frameworks web funcionales: 150 horas.
	\item Análisis de otros frameworks web basados en DSL: 100 horas.
	\item Conceptualización de las características deseables en un framework web funcional basado en DSL: 100 horas.
	\item Desarrollo de la solución propuesta a partir del análisis desarrollado: 250 horas.
	\item Desarrollo de una aplicación de referencia utilizando la solución propuesta: 150 horas.
	\item Análisis finales, conclusiones, futuras líneas de investigación: 50 horas.
\end{enumerate}

\addcontentsline{toc}{section}{Glosario}
\printglossaries

\addcontentsline{toc}{section}{Referencias}
\bibliographystyle{plain}
\bibliography{propuesta-tesis}

\end{document}
